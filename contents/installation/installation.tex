\expandafter\ifx\csname ifdraft\endcsname\relax
    %!TEX encoding = UTF-8
% +++
% latex = "uplatex"
% +++
\documentclass[uplatex,dvipdfmx,b5j,openany]{jsbook}
\usepackage{graphicx}

\usepackage{siunitx}		%for use si unit
\usepackage{url}			%for display URL
\usepackage{here}			%for use figure here
\usepackage{tikz}			%for use TikZ package
\usepackage{pgfplots}		%for use PGFplots
\usepackage{dcolumn}		%for use significant figures in the table
\usepackage{csvsimple}		%for import csv files
\usepackage[RPvoltages,americanresistors,americaninductors,europeanvoltage,americancurrents]{circuitikz}
\usepackage[noto]{pxchfon}	%for use Noto fonts

% To display TikZ logo
\usepackage{bxtexlogo}
\bxtexlogoimport{TikZ}

\usepackage{wrapfig}
\usepackage[top=1.5cm, bottom=1.5cm, left=2.5cm, right=2cm]{geometry}

% minted & mdframed settings
\usepackage[cache=false]{minted}
\usepackage[xcolor,framemethod=tikz,usetwoside=false]{mdframed}
\usetikzlibrary{shadows}
% \mdfsetup{frametitlealignment=\center}
\mdfdefinestyle{bash}{linecolor=black,linewidth=0.5pt}
\mdfdefinestyle{shadow}{linewidth=0pt,backgroundcolor=black!15}
\mdfdefinestyle{column}{linewidth=1pt,roundcorner=10pt,backgroundcolor=black!10}

\newenvironment{column}[1]
  {\mdfsetup{
    frametitle={\colorbox{white}{\space#1\space}},
    innertopmargin=2pt,
	innerbottommargin=10pt,
    frametitleaboveskip=-\ht\strutbox,
	roundcorner=5pt,
	backgroundcolor=black!10
    }
  \begin{mdframed}%
  }
  {\end{mdframed}}

\tikzset{% tikz style set
  	pointtype triangle/.style={mark=triangle*,mark size=4pt},
  	every mark/.style={fill=white,solid},
  	south west label/.style={
		matrix,matrix of nodes,
		anchor=south west,at={(rel axis cs:0.01,0.01)},
		nodes={anchor=west,inner sep=0},
  	},
}

\pgfplotsset{% graph style set
    table/col sep=comma, % Use CSV files
  	compat=1.12,
  	major tick length=0.2cm,
  	minor tick length=0.1cm,
  	every axis/.style={semithick},
  	tick style={semithick,black},
  	legend cell align=left,
  	legend image code/.code={%
		\draw[mark repeat=2,mark phase=2,#1]
	  	plot coordinates {(0cm,0cm) (0.5cm,0cm) (1.0cm,0cm)};
  	},
  	log number format basis/.code 2 args={
	\pgfmathsetmacro\e{#2}
	\pgfmathparse{#2==0}\ifnum\pgfmathresult>0{1}\else
	\pgfmathparse{#2==1}\ifnum\pgfmathresult>0{10}\else
	{$#1^{\pgfmathprintnumber{\e}}$}\fi\fi},
}

% macros
\newcommand{\logoLaTeX}{{\rm \textbf \LaTeX}\hspace{0zw}}
\newcommand{\logoCiTikZ}{Circui{\TikZ}}
	\begin{document}
\fi

\chapter{\logoLaTeX 使用環境の構築}
	\logoCiTikZ は\LaTeX の拡張機能ですので、\LaTeX の使用環境を整える必要があります。
	本書の目的は\logoCiTikZ を使用することでありますので、
	本格的な環境構築は行わず、OverleafというWeb上で\LaTeX を使用することができるサービスを使用します。

	\section{Overleafの設定}
		まず、Overleafをブラウザで開きます。URLは以下の通りです。
		\begin{mdframed}[style=shadow]
			\url{https://www.overleaf.com}
		\end{mdframed}
		すると以下のようなページが開きます。
		\begin{figure}[H]
			\centering
			\includegraphics[width=\textwidth]{figure/overleaf-page-top.png}
		\end{figure}
		右上のRegisterをクリックし、会員登録を行います。

		\newpage
		\begin{wrapfigure}{r}[0pt]{0.6\textwidth}
			\centering

			\includegraphics[width=0.6\textwidth]{figure/overleaf-page-createproject.png}

			\vspace{3mm}

			\begin{flushright}
				\includegraphics[width=0.55\textwidth]{figure/overleaf-page-projectname.png}
			\end{flushright}
		\end{wrapfigure}
		会員登録が完了すると以下のようなページが開くので、
		"Cleate First Project"をクリックし、"Blank Project"を選択します。
		
		New Projectというポップアップが出てくるので、ここにProject名を入力します。
		どんな名前でも問題なく動作しますが、今回は解説のために
		\begin{mdframed}[style=shadow]
			\url{circuitikz-template}
		\end{mdframed}\vspace{-3mm}
		と入力してください。
		

\expandafter\ifx\csname ifdraft\endcsname\relax
	\end{document}
\fi