\expandafter\ifx\csname ifdraft\endcsname\relax
    %!TEX encoding = UTF-8
% +++
% latex = "uplatex"
% +++
\documentclass[uplatex,dvipdfmx,b5j,openany]{jsbook}
\usepackage{graphicx}

\usepackage{siunitx}		%for use si unit
\usepackage{url}			%for display URL
\usepackage{here}			%for use figure here
\usepackage{tikz}			%for use TikZ package
\usepackage{pgfplots}		%for use PGFplots
\usepackage{dcolumn}		%for use significant figures in the table
\usepackage{csvsimple}		%for import csv files
\usepackage[RPvoltages,americanresistors,americaninductors,europeanvoltage,americancurrents]{circuitikz}
\usepackage[noto]{pxchfon}	%for use Noto fonts

% To display TikZ logo
\usepackage{bxtexlogo}
\bxtexlogoimport{TikZ}

\usepackage{wrapfig}
\usepackage[top=1.5cm, bottom=1.5cm, left=2.5cm, right=2cm]{geometry}

% minted & mdframed settings
\usepackage[cache=false]{minted}
\usepackage[xcolor,framemethod=tikz,usetwoside=false]{mdframed}
\usetikzlibrary{shadows}
% \mdfsetup{frametitlealignment=\center}
\mdfdefinestyle{bash}{linecolor=black,linewidth=0.5pt}
\mdfdefinestyle{shadow}{linewidth=0pt,backgroundcolor=black!15}
\mdfdefinestyle{column}{linewidth=1pt,roundcorner=10pt,backgroundcolor=black!10}

\newenvironment{column}[1]
  {\mdfsetup{
    frametitle={\colorbox{white}{\space#1\space}},
    innertopmargin=2pt,
	innerbottommargin=10pt,
    frametitleaboveskip=-\ht\strutbox,
	roundcorner=5pt,
	backgroundcolor=black!10
    }
  \begin{mdframed}%
  }
  {\end{mdframed}}

\tikzset{% tikz style set
  	pointtype triangle/.style={mark=triangle*,mark size=4pt},
  	every mark/.style={fill=white,solid},
  	south west label/.style={
		matrix,matrix of nodes,
		anchor=south west,at={(rel axis cs:0.01,0.01)},
		nodes={anchor=west,inner sep=0},
  	},
}

\pgfplotsset{% graph style set
    table/col sep=comma, % Use CSV files
  	compat=1.12,
  	major tick length=0.2cm,
  	minor tick length=0.1cm,
  	every axis/.style={semithick},
  	tick style={semithick,black},
  	legend cell align=left,
  	legend image code/.code={%
		\draw[mark repeat=2,mark phase=2,#1]
	  	plot coordinates {(0cm,0cm) (0.5cm,0cm) (1.0cm,0cm)};
  	},
  	log number format basis/.code 2 args={
	\pgfmathsetmacro\e{#2}
	\pgfmathparse{#2==0}\ifnum\pgfmathresult>0{1}\else
	\pgfmathparse{#2==1}\ifnum\pgfmathresult>0{10}\else
	{$#1^{\pgfmathprintnumber{\e}}$}\fi\fi},
}

% macros
\newcommand{\logoLaTeX}{{\rm \textbf \LaTeX}\hspace{0zw}}
\newcommand{\logoCiTikZ}{Circui{\TikZ}}
	\graphicspath{{./figure/}}
\begin{document}
\fi

\chapter{\logoLaTeX 使用環境の構築}
	\logoCiTikZ は\LaTeX の拡張機能ですので、\LaTeX の使用環境を整える必要があります。
	本書の目的は\logoCiTikZ を使用することでありますので、
	本格的な環境構築は行わず、OverleafというWeb上で\LaTeX を使用することができるサービスを使用します。
	
	
	\section{Overleafへのログイン}
		まず、Overleafをブラウザで開きます。URLは以下の通りです。
		\begin{mdframed}[style=shadow]
			\url{https://www.overleaf.com}
		\end{mdframed}\vspace{-3mm}
		すると以下のようなページが開きます。
		\begin{figure}[H]
			\centering
			\includegraphics[width=\textwidth]{overleaf-page-top.png}
		\end{figure}
		右上のRegisterをクリックし、会員登録を行います。
		
		\newpage
	
	\section{テンプレートの作成}
		\noindent
		\begin{minipage}{0.4\hsize}%\parindent=1zw
			会員登録が完了すると右のようなページが開くので、
			"Cleate First Project"をクリックし、"Blank Project"を選択します。
			右の図のようなポップアップが出てくるので、ここにProject名を入力します。
			どんな名前でも動作しますが、今回は解説のために
			\begin{mdframed}[style=shadow]
				\url{circuitikz-template}
			\end{mdframed}\vspace{-3mm}
			と入力して、Createをクリックしてください。
		\end{minipage}\hfill
		\begin{minipage}{0.6\hsize}
			\begin{flushright}
				\includegraphics[width=0.95\textwidth]{overleaf-page-createproject.png}\\
				\vspace{3mm}
				\includegraphics[width=0.95\textwidth]{overleaf-page-projectname.png}
			\end{flushright}
		\end{minipage}
		
		\begin{figure}[H]
			\centering
			\includegraphics[width=\textwidth]{overleaf-editer-initial.png}
		\end{figure}
		
		\noindent
		\begin{minipage}{0.4\hsize}%\parindent=1zw
			すると上のような画面に遷移しますので、
			赤矢印が示すボタン"New File"をクリックしてください。
			すると右図のような画面が開くので、
			New Fileが選択されていることを確認したうえで
			\begin{mdframed}[style=shadow]
				\url{.latexmkrc}
			\end{mdframed}\vspace{-3mm}
			と入力して、Createをクリックしてください。
		\end{minipage}\hfill
		\begin{minipage}{0.6\hsize}
			\begin{flushright}
				\includegraphics[width=0.95\textwidth]{overleaf-editer-newfile.png}
			\end{flushright}
		\end{minipage}
		
		\noindent
		\begin{minipage}{0.4\hsize}
			すると左のファイル欄に\url{.latexmkrc}が追加されています。
			このファイルをクリックし、編集画面を開きます。
			編集画面を開くと、右図のように真っ白なファイルが表示されます。
			このファイルに、以下に示す設定コードを入力してください。
		\end{minipage}\hfill
		\begin{minipage}{0.6\hsize}
			\begin{flushright}
				\includegraphics[width=0.95\textwidth]{overleaf-editer-latexmkrc.png}
			\end{flushright}
		\end{minipage}
		
		\begin{mdframed}
			\begin{minted}[fontsize=\small,breaklines]{perl}
#!/usr/bin/env perl
$pdf_mode   = 3;
$latex      = 'uplatex -kanji=utf8';
$bibtex     = 'upbibtex';
$dvipdf     = 'dvipdfmx %O -o %D %S';
$makeindex  = 'mendex %O -o %D %S';
				\end{minted}
		\end{mdframed}
		
		\noindent
		\begin{minipage}{0.4\hsize}\parindent=1zw
			入力すると、右図のような状態になります。
			
			次に、左のファイル欄の\url{main.tex}を選択します。
			ファイルの中に何か記述されていますが、これを一度すべて削除し、空白の状態にします。
			空白になったファイルの中に、以下に示すコードを入力してください。
		\end{minipage}\hfill
		\begin{minipage}{0.6\hsize}
			\begin{flushright}
				\includegraphics[width=0.95\textwidth]{overleaf-text-latexmkrc.png}
			\end{flushright}
		\end{minipage}
		
		\begin{mdframed}
			\begin{minted}[fontsize=\small,breaklines]{tex}
\documentclass[uplatex,dvipdfmx]{jsarticle}
\usepackage{graphicx}

\usepackage{tikz}
\usepackage[RPvoltages]{circuitikz}
\usepackage{here}

\author{ななしたろう}
\title{CircuiTikZテンプレート}

\begin{document}
\maketitle

\section{はじめに}
    この文書はテンプレート用です。
    コピーの上ご使用ください。

\section{回路図のテスト}
    \begin{figure}[H]
        \centering
        \begin{circuitikz}[scale=1.5]
            \draw (0,0)
            to[sV,l=$FG$] (0,2);
            \draw (0,2)
            to[short] (2,2);
            \draw (2,2)
            to[R,l_=$R$] (2,1);
            \draw (2,1)
            to[L,l_=$L$] (2,0);
            \draw (2,0)
            to[short] (0,0);
        \end{circuitikz}
    \end{figure}

\end{document}
				\end{minted}
		\end{mdframed}
		
		\noindent
		\begin{minipage}{0.65\hsize}\parindent=1zw
			コードを入力したら、Overleafの左上にあるMenuボタンをクリックします。
			Menuボタンをクリックすると、右図のような画面が表示されます。
			"Compiler"のタブをクリックし"LaTeX"を選択し、元の編集画面をクリックしてもどります。
			左側のRecompileと書かれたボタンをクリックすると、コンパイルがはじまります。
			\includegraphics[width=\textwidth]{overleaf-example-pdf.png}
			しばらくすると、コンパイルが終了して下の画像のようなPDFが出力されていたら完了となります。
		\end{minipage}\hfill
		\begin{minipage}{0.35\hsize}
			\begin{flushright}
				\includegraphics[width=0.95\textwidth]{overleaf-compiler-setting.png}
			\end{flushright}
		\end{minipage}
		
		\begin{column}{もしもうまくいかなかったら}
			環境構築の手順について解説しましたが、うまくいかない場合もあると思います。
			その場合はまず、\url{.latexmkrc}や\url{main.tex}を
			正しく入力できているか確認してみてください。
			PDFからコピーアンドペーストすると正しく改行などがなされず、
			エラーの原因となっている可能性があります。
			また\url{https://3rdJCG.dev}からこれらのファイルの内容を
			そのままコピーできるのでそちらを利用すると確実です。
		\end{column}

	\section{テンプレートのコピー}
		先ほど行った作業を新しい文書を作るたびに行うのは手間なので、
		これから新しい文書を作成するときは、このテンプレートをコピーします。
		その方法を解説します。

		まず、さきほどのOverleafの画面の左上にある↑をクリックしてメニュー画面に戻ります。
		\footnote{うまく開けない場合は以下のURLから\url{https://www.overleaf.com/project}}
		すると以下のような画面が表示されるので、赤丸で示した"copy"ボタンを押下します。
		
		\includegraphics[width=0.95\textwidth]{overleaf-menu.png}
		
		すると以下のような画面に遷移するので、好きな名前を入力して"copy"をクリックします。

		\includegraphics[width=0.95\textwidth]{overleaf-menu-copy.png}

		コピーが完了すると、以下のようにコピーされたプロジェクトが表示されています。

		\includegraphics[width=0.95\textwidth]{overleaf-menu-copied.png}
		
		\expandafter\ifx\csname ifdraft\endcsname\relax
\end{document}
\fi