\expandafter\ifx\csname ifdraft\endcsname\relax
    %!TEX encoding = UTF-8
% +++
% latex = "uplatex"
% +++
\documentclass[uplatex,dvipdfmx,b5j,openany]{jsbook}
\usepackage{graphicx}

\usepackage{siunitx}		%for use si unit
\usepackage{here}			%for use figure here
\usepackage{tikz}			%for use TikZ package
\usepackage{pgfplots}		%for use PGFplots
\usepackage{dcolumn}		%for use significant figures in the table
\usepackage{csvsimple}		%for import csv files
\usepackage[RPvoltages,americanresistors,americaninductors,europeanvoltage,americancurrents]{circuitikz}
\usepackage[noto]{pxchfon}	%for use Noto fonts

% To out put TikZ logo
\usepackage{bxtexlogo}
\bxtexlogoimport{TikZ}

\usepackage{wrapfig}
\usepackage[top=1.5cm, bottom=1.5cm, left=2.5cm, right=2cm]{geometry}

% framed settings
\usepackage{framed}
\definecolor{shadecolor}{gray}{0.80}

% mdframed settings
\usepackage[xcolor,framemethod=tikz]{mdframed}
\usetikzlibrary{shadows}
\mdfdefinestyle{bash}{linecolor=black,linewidth=0.5pt}
\mdfdefinestyle{shadow}{linewidth=0pt,backgroundcolor=black!15}

\usepackage[customcolors]{hf-tikz}
\hfsetfillcolor{black!5}
\hfsetbordercolor{black!50}

\usepackage[cache=false]{minted}

\usepackage{uri}

\tikzset{% tikz style set
  	pointtype triangle/.style={mark=triangle*,mark size=4pt},
  	every mark/.style={fill=white,solid},
  	south west label/.style={
		matrix,matrix of nodes,
		anchor=south west,at={(rel axis cs:0.01,0.01)},
		nodes={anchor=west,inner sep=0},
  	},
}

\pgfplotsset{% graph style set
    table/col sep=comma, % Use CSV files
  	compat=1.12,
  	major tick length=0.2cm,
  	minor tick length=0.1cm,
  	every axis/.style={semithick},
  	tick style={semithick,black},
  	legend cell align=left,
  	legend image code/.code={%
		\draw[mark repeat=2,mark phase=2,#1]
	  	plot coordinates {(0cm,0cm) (0.5cm,0cm) (1.0cm,0cm)};
  	},
  	log number format basis/.code 2 args={
	\pgfmathsetmacro\e{#2}
	\pgfmathparse{#2==0}\ifnum\pgfmathresult>0{1}\else
	\pgfmathparse{#2==1}\ifnum\pgfmathresult>0{10}\else
	{$#1^{\pgfmathprintnumber{\e}}$}\fi\fi},
}

% macros
\newcommand{\logoLaTeX}{{\rm \textbf \LaTeX}\hspace{0zw}}
	\graphicspath{{./figure/}}
\begin{document}
\fi

\chapter{\logoLaTeX の基本}
	前章では、\LaTeX の使用環境の構築方法について解説しました。
	この章では\LaTeX の基本的な使用方法について解説していきます。

	\section{\logoLaTeX 文書の基本構成}
		\LaTeX 文書は、\url{\}からはじまる単語は命令とよばれ、特別な指示をコンパイラに与えます。
		命令には2つの形式があり、
		\url{\hogehoge}のように1つの単語で完結する命令と、
		\url{\begin{hoge}}ではじまり\url{\end{hoge}}で終わる命令があります。
		とくに\url{\begin{hoge}}と\url{\end{hoge}}の間を"hoge環境"のように呼称します。
        また他に特別な記号として\url以降の記述は改行されるまで一切の内容を無視します。
        この機能をコメントアウトと呼称します。
        この機能を使うと、メモ書きなどをしておくことができます。
        \begin{mdframed}
			\begin{minted}[fontsize=\small,breaklines]{tex}
\hogehoge % このようにコメントを書くことができます

\begin{hoge} % これはhoge環境のはじまり
    あいうえお
\end{hoge} % これはhoge環境のおわり
			\end{minted}
		\end{mdframed}
        
		また、\LaTeX 文書は2つの部分で構成されます。
		1つ目がプリアンブルと呼ばれる部分で、その文書の設定などを行います。
		2つ目が本文と呼ばれる部分で、文書の本文や文書の描画命令そのものを記述します。
        \begin{mdframed}
			\begin{minted}[fontsize=\small,breaklines]{tex}
\documentclass[uplatex,dvipdfmx]{jsarticle}
\usepackage{graphicx}
︙
︙ % ここがプリアンブル
︙
\begin{document}
    ︙
    ︙ % ここが本文
    ︙
\end{document}
			\end{minted}
		\end{mdframed}

		前章でコピーしたテンプレートを開いて、実際の\LaTeX 文書を見てみましょう。
		この文書では、\url{\begin{document}}より上の部分の1から10行目の間がプリアンブルで、
		\url{\begin{document}}から\url{\end{document}}の11行目から35行目の間が本文です。
        
        次に、それぞれの命令について見てみましょう。
        例えば、\url{\maketitle}という命令は本文の前にタイトルを出力する命令です。
        また\url{\begin{figure}}と\url{\end{figure}}の間はfigure環境と呼ばれ、
        この環境内の記述内容は図と認識されます。
        またfigure環境内に\url{\begin{circuitikz}}と\url{\end{circuitikz}}があるように、
        環境はいわゆる"入れ子構造"にすることができます。


		\expandafter\ifx\csname ifdraft\endcsname\relax
		\end{document}
		\fi